\begin{table}[h]
\begin{center}
\begin{minipage}{\textwidth}
\caption{Example of a lengthy table which is set to full textwidth}\label{tab2}
\begin{tabular*}{\textwidth}{@{\extracolsep{\fill}}lcccccc@{\extracolsep{\fill}}}
\toprule%
& \multicolumn{3}{@{}c@{}}{Element 1\footnotemark[1]} & \multicolumn{3}{@{}c@{}}{Element 2\footnotemark[2]} \\\cmidrule{2-4}\cmidrule{5-7}%
Project & Energy & $\sigma_{calc}$ & $\sigma_{expt}$ & Energy & $\sigma_{calc}$ & $\sigma_{expt}$ \\
\midrule
Element 3  & 990 A & 1168 & $1547\pm12$ & 780 A & 1166 & $1239\pm100$\\
Element 4  & 500 A & 961  & $922\pm10$  & 900 A & 1268 & $1092\pm40$\\
\botrule
\end{tabular*}
\footnotetext{Note: This is an example of table footnote. This is an example of table footnote this is an example of table footnote this is an example of~table footnote this is an example of table footnote.}
\footnotetext[1]{Example for a first table footnote.}
\footnotetext[2]{Example for a second table footnote.}
\end{minipage}
\end{center}
\end{table}

In case of double column layout, tables which do not fit in single column width should be set to full text width. For this, you need to use \verb+\begin{table*}+ \verb+...+ \verb+\end{table*}+ instead of \verb+\begin{table}+ \verb+...+ \verb+\end{table}+ environment. Lengthy tables which do not fit in textwidth should be set as rotated table. For this, you need to use \verb+\begin{sidewaystable}+ \verb+...+ \verb+\end{sidewaystable}+ instead of \verb+\begin{table*}+ \verb+...+ \verb+\end{table*}+ environment. This environment puts tables rotated to single column width. For tables rotated to double column width, use \verb+\begin{sidewaystable*}+ \verb+...+ \verb+\end{sidewaystable*}+.

\begin{sidewaystable}
\sidewaystablefn%
\begin{center}
\begin{minipage}{\textheight}
\caption{Tables which are too long to fit, should be written using the ``sidewaystable'' environment as shown here}\label{tab3}
\begin{tabular*}{\textheight}{@{\extracolsep{\fill}}lcccccc@{\extracolsep{\fill}}}
\toprule%
& \multicolumn{3}{@{}c@{}}{Element 1\footnotemark[1]}& \multicolumn{3}{@{}c@{}}{Element\footnotemark[2]} \\\cmidrule{2-4}\cmidrule{5-7}%
Projectile & Energy	& $\sigma_{calc}$ & $\sigma_{expt}$ & Energy & $\sigma_{calc}$ & $\sigma_{expt}$ \\
\midrule
Element 3 & 990 A & 1168 & $1547\pm12$ & 780 A & 1166 & $1239\pm100$ \\
Element 4 & 500 A & 961  & $922\pm10$  & 900 A & 1268 & $1092\pm40$ \\
Element 5 & 990 A & 1168 & $1547\pm12$ & 780 A & 1166 & $1239\pm100$ \\
Element 6 & 500 A & 961  & $922\pm10$  & 900 A & 1268 & $1092\pm40$ \\
\botrule
\end{tabular*}
\footnotetext{Note: This is an example of table footnote this is an example of table footnote this is an example of table footnote this is an example of~table footnote this is an example of table footnote.}
\footnotetext[1]{This is an example of table footnote.}
\end{minipage}
\end{center}
\end{sidewaystable}